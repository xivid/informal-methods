\documentclass[a4paper, 11pt]{article}
\usepackage{comment} % enables the use of multi-line comments (\ifx \fi) 
\usepackage{lipsum} %This package just generates Lorem Ipsum filler text. 
\usepackage{fullpage} % changes the margin
\usepackage{amsmath}

\begin{document}

\noindent
\large\textbf{Task 1} \hfill \textbf{17-941-998}

\section*{Third Procedure: \texttt{transfer()}}

Denote the length, head and tail pointers of the two buffers with \texttt{NB1,h1,t1,NB2,h1,t2}, respectively. The variables in \texttt{push()} and \texttt{pop()} also need replacing accordingly.

\begin{verbatim}
push():
  wait until t1 - h1 < NB1
  B1[t1 mod NB1] := I[x]
  x := x + 1
  t1 := t1 + 1

transfer():
  wait until h1 < t1 and t2 - h2 < NB2
  B2[t2 mod NB2] := B1[h1 mod NB1]
  h1 := h1 + 1
  t2 := t2 + 1

pop():
  wait until h2 < t2
  O[y]:= B2[h2 mod NB2]
  h2 := h2 + 1
  y := y + 1
\end{verbatim}

\section*{Interpretations}
To write an invariant of the form:
$$ S_O+S_{B_2O}+S_{B_2}+S_{B_1B_2}+S_{B_1}+S_{IB_1}+S_I = K, $$
the following interpretations are needed:
\begin{flalign*}
\qquad\qquad\qquad\qquad S_I &= I[x,N)
&S_{IB_1} &= B_1[t_1\ \text{upto}\ x) \qquad\qquad\qquad\qquad\\
S_{B_1} &= B_1[h_1\ \text{upto}\ t_1)
&S_{B_1B_2} &= B_2[t_2\ \text{upto}\ h_1) \\
S_{B_2} &= B_2[h_2\ \text{upto}\ t_2) 
&S_{B_2O} &= O[y\ \text{upto}\ h_2) \\
S_O &= O[0,y]
\end{flalign*}

\section*{Invariant}

The updated invariant is 
$$ S_O+S_{B_2O}+S_{B_2}+S_{B_1B_2}+S_{B_1}+S_{IB_1}+S_I = K $$
with side conditions
$$ h_1 \leq t_1 \land t_1-h_1 \leq N_{B_1} \land h_2 \leq t_2 \land t_2-h_2 \leq N_{B_2}. $$

\subsection*{Local Correctness}

\noindent For \texttt{push()}, a local invariant $S: x \leq N$ is added, $t_1=x \land x<N$ are preconditions, and $t_1=x$ is a postcondition:
\begin{flalign*}
& \triangleright S_O+S_{B_2O}+S_{B_2}+S_{B_1B_2}+S_{B_1}+S_{IB_1}+S_I=K \land x<N \land & \\ 
& \quad h_1 \leq t_1 \land t_1-h_1 \leq N_{B_1} \land h_2 \leq t_2 \land t_2-h_2 \leq N_{B_2}  &\\
& \triangleright t_1=x &\\
% 1
& \textbf{wait until } t_1-h_1<N_{B_1} & \\
% 
& \triangleright S_O+S_{B_2O}+S_{B_2}+S_{B_1B_2}+B_1[h_1\ \text{upto}\ t_1)+[\ ]+(I[x]+I[x+1, N))=K \land x<N \land & \\ 
& \quad h_1 \leq t_1 \land t_1-h_1 < N_{B_1} \land h_2 \leq t_2 \land t_2-h_2 \leq N_{B_2}  &\\
& \triangleright t_1=x &\\
% 2
& \mathbf{B_1}[t_1\ \text{mod} \ N_{B_1}] := I[x] & \\
%
& \triangleright S_O+S_{B_2O}+S_{B_2}+S_{B_1B_2}+B_1[h_1\ \text{upto}\ t_1)+[\ ]+(B_1[t_1\ \text{mod} \ N_{B_1}]+I[x+1, N))=K \land  & \\ 
& \quad x<N \land h_1 \leq t_1 \land t_1-h_1 < N_{B_1} \land h_2 \leq t_2 \land t_2-h_2 \leq N_{B_2}  &\\
& \triangleright t_1=x &\\
% 3
& \mathbf{x} := x + 1 & \\
%
& \triangleright S_O+S_{B_2O}+S_{B_2}+S_{B_1B_2}+B_1[h_1\ \text{upto}\ t_1)+B_1[t_1\ \text{mod} \ N_{B_1}]+I[x, N)=K \land x \leq N \land & \\ 
& \quad h_1 \leq t_1 \land t_1-h_1 < N_{B_1} \land h_2 \leq t_2 \land t_2-h_2 \leq N_{B_2}  &\\
& \triangleright t_1+1=x &\\
% 4
& \mathbf{t_1} := t_1 + 1 & \\
%
& \triangleright S_O+S_{B_2O}+S_{B_2}+S_{B_1B_2}+B_1[h_1\ \text{upto}\ t_1)+[\ ]+I[x, N)=K \land x \leq N \land & \\ 
& \quad h_1 \leq t_1 \land t_1-h_1 \leq N_{B_1} \land h_2 \leq t_2 \land t_2-h_2 \leq N_{B_2}  &\\
& \triangleright t_1=x &
% 5
\end{flalign*}

\noindent For \texttt{pop()}, a local invariant $R: y \leq N$ is added, $y=h_2 \land y<N$ are preconditions, and $y=h_2$ is a postcondition:
\begin{flalign*}
& \triangleright S_O+S_{B_2O}+S_{B_2}+S_{B_1B_2}+S_{B_1}+S_{IB_1}+S_I=K \land y<N \land & \\ 
& \quad h_1 \leq t_1 \land t_1-h_1 \leq N_{B_1} \land h_2 \leq t_2 \land t_2-h_2 \leq N_{B_2}  &\\
& \triangleright y=h_2 &\\
% 1
& \textbf{wait until } h_2<t_2 & \\
% 
& \triangleright O[0,y)+[\ ]+(B_2[h_2\ \text{mod}\ N_{B_2}]+B_2[h_2+1\ \text{upto}\ t_2)) +S_{B_1B_2}+S_{B_1}+S_{IB_1}+S_I =K \land & \\ 
& \quad y<N \land h_1 \leq t_1 \land t_1-h_1 \leq N_{B_1} \land h_2 < t_2 \land t_2-h_2 \leq N_{B_2}  &\\
& \triangleright y=h_2 &\\
% 2
& \mathbf{O}[y] := B_2[h_2\ \text{mod}\ N_{B_2}] & \\
%
& \triangleright O[0,y)+[\ ]+(O[y]+B_2[h_2+1\ \text{upto}\ t_2)) +S_{B_1B_2}+S_{B_1}+S_{IB_1}+S_I =K \land y<N \land  & \\ 
& \quad h_1 \leq t_1 \land t_1-h_1 \leq N_{B_1} \land h_2 < t_2 \land t_2-h_2 \leq N_{B_2}  &\\
& \triangleright y=h_2 &\\
% 3
& \mathbf{h_2} := h_2 + 1 & \\
%
& \triangleright O[0,y)+O[y]+B_2[h_2\ \text{upto}\ t_2) +S_{B_1B_2}+S_{B_1}+S_{IB_1}+S_I =K \land y < N \land & \\ 
& \quad h_1 \leq t_1 \land t_1-h_1 \leq N_{B_1} \land h_2 \leq t_2 \land t_2-h_2 \leq N_{B_2}  &\\
& \triangleright y+1=h_2 &
% 4
\end{flalign*}
\begin{flalign*}
& \mathbf{y} := y + 1 & \\
%
& \triangleright O[0,y)+[\ ]+B_2[h_2\ \text{upto}\ t_2) +S_{B_1B_2}+S_{B_1}+S_{IB_1}+S_I =K \land y \leq N \land & \\ 
& \quad h_1 \leq t_1 \land t_1-h_1 \leq N_{B_1} \land h_2 \leq t_2 \land t_2-h_2 \leq N_{B_2}  &\\
& \triangleright y=h_2 &
% 5
\end{flalign*}

\noindent Finally, for \texttt{transfer()}, $t_2=h_1$ is both a precondition and a postcondition:
\begin{flalign*}
& \triangleright S_O+S_{B_2O}+S_{B_2}+S_{B_1B_2}+S_{B_1}+S_{IB_1}+S_I=K \land & \\ 
& \quad h_1 \leq t_1 \land t_1-h_1 \leq N_{B_1} \land h_2 \leq t_2 \land t_2-h_2 \leq N_{B_2}  &\\
& \triangleright t_2=h_1 &\\
% 1
& \textbf{wait until } h_1<t_1\ \textbf{and}\ t_2-h_2<N_{B_2} & \\
% 
& \triangleright S_O+S_{B_2O}+B_2[h_2\ \text{upto}\ t_2)+[\ ]+(B_1[h_1\ \text{mod}\ N_{B_1}]+B_1[h_1+1\ \text{upto}\ t_1)) +S_{IB_1}+S_I =K \land & \\ 
& \quad h_1 < t_1 \land t_1-h_1 \leq N_{B_1} \land h_2 \leq t_2 \land t_2-h_2 < N_{B_2}  &\\
& \triangleright t_2=h_1 &\\
% 2
& \mathbf{B_2}[t_2\ \text{mod}\ N_{B_2}] := B_1[h_1\ \text{mod}\ N_{B_1}] & \\
%
& \triangleright S_O+S_{B_2O}+B_2[h_2\ \text{upto}\ t_2)+[\ ]+(B_2[t_2\ \text{mod}\ N_{B_2}]+B_1[h_1+1\ \text{upto}\ t_1)) +S_{IB_1}+S_I =K \land & \\ 
& \quad h_1 < t_1 \land t_1-h_1 \leq N_{B_1} \land h_2 \leq t_2 \land t_2-h_2 < N_{B_2}  &\\
& \triangleright t_2=h_1 &\\
% 3
& \mathbf{h_1} := h_1 + 1 & \\
%
& \triangleright S_O+S_{B_2O}+B_2[h_2\ \text{upto}\ t_2)+B_2[t_2\ \text{mod}\ N_{B_2}]+B_1[h_1\ \text{upto}\ t_1) +S_{IB_1}+S_I =K \land & \\ 
& \quad h_1 \leq t_1 \land t_1-h_1 \leq N_{B_1} \land h_2 \leq t_2 \land t_2-h_2 < N_{B_2}  &\\
& \triangleright t_2+1=h_1 &\\
% 4
& \mathbf{t_2} := t_2 + 1 & \\
%
& \triangleright S_O+S_{B_2O}+B_2[h_2\ \text{upto}\ t_2)+[\ ]+B_1[h_1\ \text{upto}\ t_1) +S_{IB_1}+S_I =K \land & \\ 
& \quad h_1 \leq t_1 \land t_1-h_1 \leq N_{B_1} \land h_2 \leq t_2 \land t_2-h_2 \leq N_{B_2}  &\\
& \triangleright t_2=h_1 &
% 5
\end{flalign*}

For all three functions, the statements at every point are at least as strong as the invariant, so the local correctness is maintained.

\subsection*{Noninterference Proof}
The noninterference proof is still easy. All assertions are a conjunction of predicates of 5 forms:
\begin{itemize}
    \item $S_O+S_{B_2O}+S_{B_2}+S_{B_1B_2}+S_{B_1}+S_{IB_1}+S_I = K$, which is included in the invariant, and proved to hold everywhere as long as other conjuncts hold.
    \item $x<N, x \leq N, y < N, y \leq N$, which only involve local variables $x,y$ and the constant $N$, so interference is impossible.
    \item $t_1=x, t_1+1=x, y=h_2, y+1=h_2, t_2=h_1, t_2+1=h_1$, which involve shared variables $t_1,h_2,t_2,h_1$, but they are only ever updated by the concerned process itself: only \textbf{S} touches $t_1$, only \textbf{R} touches $h_2$, and only \textbf{T} (the process calling \texttt{transfer()}) touches $t_2$ and $h_1$.
    \item $h_1 \leq t_1, t_1 - h_1 \leq N_{B_1}, h_2 \leq t_2, t_2 - h_2 \leq N_{B_2}$, which are also parts of the invariant, so already covered.
\end{itemize}

This leaves only $t_1-h_1<N_{B_1}$, $h_2<t_2$, $h_1<t_1$ and $t_2-h_2<N_{B_2}$ as predicates that might suffer interference:

\begin{enumerate}
    \item $t_1-h_1<N_{B_1}$ only appears in \textbf{S}, and the only statement outside \textbf{S} that might invalidate it is $h_1:=h_1+1$ in \textbf{T}, with the following annotation:
\begin{flalign*}
& \triangleright t_1-h_1<N_{B_1} &\\
% 
& \mathbf{h_1} := h_1 + 1 & \\
%
& \triangleright t_1-h_1<N_{B_1} &
\end{flalign*}
\textbf{T} only makes $S_{B_1}$ shorter, so if it was not full, it still is not.
    
    \item $h_2<t_2$ only appears in \textbf{R}, and the only statement outside \textbf{R} that might invalidate it is $t_2:=t_2+1$ in \textbf{T}, with the following annotation:
\begin{flalign*}
& \triangleright h_2<t_2 &\\
% 
& \mathbf{t_2} := t_2 + 1 & \\
%
& \triangleright h_2<t_2 &
\end{flalign*}
\textbf{T} only makes $S_{B_2}$ longer, so if it was not empty, it still is not.

    \item $h_1<t_1$ only appears in \textbf{T}, and the only statement outside \textbf{T} that might invalidate it is $t_1:=t_1+1$ in \textbf{S}, with the following annotation:
\begin{flalign*}
& \triangleright h_1<t_1 &\\
% 
& \mathbf{t_1} := t_1 + 1 & \\
%
& \triangleright h_1<t_1 &
\end{flalign*}
\textbf{S} only makes $S_{B_1}$ longer, so if it was not empty, it still is not.

    \item $t_2-h_2<N_{B_2}$ only appears in \textbf{T}, and the only statement outside \textbf{T} that might invalidate it is $h_2:=h_2+1$ in \textbf{R}, with the following annotation:
\begin{flalign*}
& \triangleright t_2-h_2<N_{B_2} &\\
% 
& \mathbf{h_2} := h_2 + 1 & \\
%
& \triangleright t_2-h_2<N_{B_2} &
\end{flalign*}
\textbf{R} only makes $S_{B_2}$ shorter, so if it was not full, it still is not.
    
\end{enumerate}


% to comment sections out, use the command \ifx and \fi. Use this technique when writing your pre lab. For example, to comment something out I would do:
%  \ifx
%	\begin{itemize}
%		\item item1
%		\item item2
%	\end{itemize}	
%  \fi

\section*{Partial Correctness}
From initialization $x,y,h_1,t_1,h_2,t_2:=0,0,0,0,0,0$ and the extended invariant, we know that $ [\ ] + [\ ] + [\ ] + [\ ] + [\ ] + [\ ] + I[0,N) = K $, so $ K = I[0,N) $.

The extended invariant plus the termination condition ($y=N$) for \textbf{R} implies that 
$$ O[0,N) + S_{B_2O}+S_{B_2}+S_{B_1B_2}+S_{B_1}+S_{IB_1}+S_I = K = I[0, N),$$ which implies
$$ N + |S_{B_2O}| + |S_{B_2}| + |S_{B_1B_2}| + |S_{B_1}| + |S_{IB_1}| + |S_I| = |I[0, N)| = N, $$ therefore
$$ |S_{B_2O}| = |S_{B_2}| = |S_{B_1B_2}| = |S_{B_1}| = |S_{IB_1}| = |S_I| = 0,$$ thus
$$ O[0,N) = I[0,N).$$
\end{document}
